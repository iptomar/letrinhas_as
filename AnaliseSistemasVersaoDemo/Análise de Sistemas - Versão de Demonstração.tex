
% !TeX spellcheck = pt_BR
% !TeX encoding = UTF-8
\documentclass[a4paper,titlepage]{article}

\usepackage[utf8]{inputenc}
\usepackage[portuguese]{babel}
\usepackage{hyperref}
\usepackage{listingsutf8}
\usepackage{float}
\usepackage{color}
\usepackage[usenames,dvipsnames]{xcolor}
\usepackage{graphicx}
% \usepackage{indentfirst}
\usepackage{epstopdf}
\usepackage{eurosym}
\usepackage{tabularx}
\usepackage{svg}
%\usepackage{mathtools}
%\usepackage{amsmath}

% Uma forma de usar caracteres especiais nos ficheiros de código-fonte.
\lstset{literate=
	{á}{{\'a}}1 {é}{{\'e}}1 {í}{{\'i}}1 {ó}{{\'o}}1 {ú}{{\'u}}1
	{Á}{{\'A}}1 {É}{{\'E}}1 {Í}{{\'I}}1 {Ó}{{\'O}}1 {Ú}{{\'U}}1
	{à}{{\`a}}1 {è}{{\'e}}1 {ì}{{\`i}}1 {ò}{{\`o}}1 {ò}{{\`u}}1
	{À}{{\`A}}1 {È}{{\'E}}1 {Ì}{{\`I}}1 {Ò}{{\`O}}1 {Ò}{{\`U}}1
	{ä}{{\"a}}1 {ë}{{\"e}}1 {ï}{{\"i}}1 {ö}{{\"o}}1 {ü}{{\"u}}1
	{Ä}{{\"A}}1 {Ë}{{\"E}}1 {Ï}{{\"I}}1 {Ö}{{\"O}}1 {Ü}{{\"U}}1
	{â}{{\^a}}1 {ê}{{\^e}}1 {î}{{\^i}}1 {ô}{{\^o}}1 {û}{{\^u}}1
	{Â}{{\^A}}1 {Ê}{{\^E}}1 {Î}{{\^I}}1 {Ô}{{\^O}}1 {Û}{{\^U}}1
	{œ}{{\oe}}1 {Œ}{{\OE}}1 {æ}{{\ae}}1 {Æ}{{\AE}}1 {ß}{{\ss}}1
	{ç}{{\c c}}1 {Ç}{{\c C}}1 {ø}{{\o}}1 {å}{{\r a}}1 {Å}{{\r A}}1
	{õ}{{\~{o}}}1
	{ã}{{\~a}}1
	{€}{{\EUR}}1 {£}{{\pounds}}1
}

\hypersetup{
    colorlinks = false,
    hidelinks
}

\lstdefinestyle{masm} {
	belowcaptionskip=1\baselineskip,
	breaklines=true,
	breakatwhitespace=true,
	xleftmargin=\parindent,
	extendedchars=true,
	language=[x86masm]{assembler},
	captionpos=b,
	numbersep=5pt,
	basicstyle=\ttfamily\footnotesize,
	keywordstyle=\bfseries\color{blue},
	commentstyle=\itshape\color{CadetBlue},
	identifierstyle=\color{black},
	frame=single,
	tabsize=2
}
	\title{Análise de Sistemas\\ Versão - Demo  \\ Letrinhas}
\author{Grupo de Análise de Sistemas}



\begin{document}

	\maketitle %titulo
	\newpage
	\tableofcontents %indice
	\newpage %pagina nova

	 \section{Introdução}%resumo - introdução
		\subsection{Visão Geral do Sistema}
		 O sistema descrito nesta análise é uma aplicação para tablet e respetivo sistema de informação. O sistema permite a avaliação da fluência na leitura dos alunos. 
		 	\subsubsection{Aplicação para tablet}
			Criação de uma aplicação em tablets de sistema operativo android, que permita a avaliação da fluência na leitura dos alunos.  
			
			\subsubsection{Sistema de informação}
			O sistema de informação fornece suporte à aplicação em android, recorrendo a uma base de dados e formulários web.
				
		\subsection{Objetivos}
			O letrinhas tem como objetivo facilitar a avaliação dos alunos no processo de aprendizagem da leitura.\\
			
			 Recorrendo a dispositivos móveis de sistema operativo android o aluno terá acesso a uma aplicação onde estarão disponiveis vários textos para leitura com diferentes graus de dificuldade, que serão posteriormente submetidos por indicação do professor.\\
			 
			  O sistema de informação será apenas disponibilizado aos professores e tem como objectivo o fácil acesso aos textos lidos e submetidos pelos alunos, para que desta forma possam ser avaliados de forma eficiente e eficaz. Guardando assim um registo por aluno.
			
		\newpage
		\subsection{Cliente}
				Os nossos serviços foram requisitados pelo agrupamento de escolas Artur Gonçalves. A ponte de comunicação entre as duas partes foi feita pelos docentes da disciplina de projeto de sistemas de informação, professor António Manso e Pedro Dias. 
				
				\subsection{Fontes e Material de Referência}
				Para melhor compreensão do conceito, utilizámos como material de referência os seguintes.
				\subsubsection{Aplicações para gravação de voz} 
					\begin{itemize}
						\item eRecorder (Aplicação para android);
						\item Smart Voice Recorder (Aplicação para Android);
						\item Voice Recorder Pro. (Aplicação para Android);		
					\end{itemize} 
				\subsubsection{Aplicações de Tradução de voz para Texto}
					\begin{itemize}
						\item ListNote Fala-para-texto Notas (Aplicação para Android); 
						\item Speach to text (Aplicação para Android);
						\item Text to speach - Voice to text (Aplicação para Android);
					\end{itemize}
				\subsubsection{Software Open Source}
					\begin{itemize}
						\item Mi sound recorder (Aplicação para Android);
						\item Auphonic Software (Aplicação para Android \& Iphone);
						\item Android Voice Recognition Tutorial;
						\item Rehearsal Assistant (Aplicação para Android);
					\end{itemize}
				
				\subsection{Glossário}
				\begin{tabular}{|c|p{8cm}|}
				\hline Android & Sistema operativo para dispositivos móveis \\ 
				\hline Open Source  & Software Informático que respeita as quatro liberdades definidas pela Free Software Foundation \\ 
				\hline 
				\end{tabular} 
				
				\newpage
				
				\section{Modelo de Casos de Utilização}
					\subsection{Atores}
						\subsubsection{Aplicação}
							\begin{tabular}{|c|p{9cm}|}
								\hline Ator & Descrição \\ 
								\hline  Aluno  & Ator do sistema que interage directamente com a aplicação realizando os testes propostos. \\ 
								\hline  Professor  & Quem distribui e supervisiona o teste realizado pelo aluno. \\ 
								\hline 
							\end{tabular} 
						\subsubsection{Sistema de Informação}
							\begin{tabular}{|c|p{8cm}|}
								\hline Ator & Descrição \\ 
								\hline  Administrador  & Ator responsável por criar contas de professor.    \\ 
								\hline  Professor  & Ator responsável por manipular o conteúdo da base de dados, criar e consultar testes de avaliação de leitura. \\ 
								\hline 
							\end{tabular} 			

				
				
		
		\end{document}